\chapter{Metode}\label{chap:Metode}

%%%%%%%%%% Vision system %%%%%%%%%%
\section{Vision system}
Overordnet beskrivelse af vision systemet.

%%%%% MATLAB - Color Thresholder %%%%%
\subsection{MATLAB - Color Thresholder}
MATLAB's  applikation Color Thresholder er blevet brugt til at bestemme, hvilket color space og hvilke farvegrænseværdier, der er mest optimale til at identificere de farvede klodser med webcamet.

I Color Thresholder indlæses et billede, taget med webcamet, hvorpå en klods af hver farve er repræsenteret.
Valget af color space baseres på hvor klodserne adskiller sig mest fra bordpladen.
På figur \vref{fig:ColorThresholderColorSpace} ses det, at der i color spacet HSV er størst kontrast mellem klodser og bordplade.
Det vælges derfor at arbejde i color spacet HSV.

\figur{1}{ColorThresholderColorSpace.pdf}{MATLAB's applikation Color Thresholder til identifikation af optimalt color space.}{fig:ColorThresholderColorSpace}

Color Thresholder bruges derefter til at finde grænseværdierne til hhv. H, S og V kanalerne, til identifikation af de forskelligt farvede klodser.
Figur \vref{fig:ColorThresholderBlue} er et eksempel på grænseværdierne til identifikation af den blå klods.

\figur{1}{ColorThresholderBlue.pdf}{Identifikation af blå klods i MATLAB's applikation Color Thresholder i color space HSV}{fig:ColorThresholderBlue}

For hver farve blev følgende grænseværdier identificeret (se bilag \vref{app:MATLABVision} for funktioner til identifikation af hver farve):
\begin{table}[H]
\centering
\begin{tabular}{l|l|l|l}
Farve	&	H			&	S			&	V\\
\hline
Blå		&	0.515-0.790	&	0.300-1.000	&	0.400-1.000\\
Rød		&	0.900-0.080	&	0.300-1.000	&	0.000-1.000\\
Grøn	&	0.200-0.415	&	0.300-1.000	&	0.000-1.000\\
Gul		&	0.115-0.210	&	0.300-1.000	&	0.400-1.000\\
\end{tabular}	
\caption{Grænseværdier til identifikation af de fire farver, blå, rød, grøn og gul. MATLAB range 0-1.}
\end{table}

Det ses på figur \vref{fig:ColorThresholderBlue}, at nogen områder uden for bordpladen ikke forsvinder helt ved denne segmentering, da de har farver, der ligger tæt på om den blå klods.
Dette ville løses ved at beskære billedet således, at kun bordpladen vises på billedet.
Derudover skal der kun identificeres klodser på højre side af billedet, hvorfor billedet beskæres yderligere, så kun det interessante område er vises.

Figur \vref{fig:ColorThresholderResult} viser resultatet på MATLAB analysen, som beskærer billedet og identificerer klodser i de fire farver, blå, rød, grøn og gul.
Se bilag \vref{app:MATLABVision} for script og funktioner brugt i analysen.
Det ses på figur \vref{fig:ColorThresholderResult}, at det vil være nødvendigt at implementerer en grænseværdi for størrelsen af klodserne, da der er små områder der ikke er forsvundet helt ved segmenteringen.

\figur{1}{ColorThresholderResult.pdf}{Resultat af MATLAB analysen}{fig:ColorThresholderResult}

%%%%% OpenCV vision %%%%%
\subsection{OpenCV vision}