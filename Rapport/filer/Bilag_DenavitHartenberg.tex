\chapter{Denavit Hartenberg}\label{app:DenavitHartenberg}

%%%%%%%%%% Coordinate frames %%%%%%%%%%
Denavit Hartenberg konventionen beskriver en fast fremgangsmåde til at tildele koordinatrammer til en robotarms led, og derefter udlede dertilhørende Denavit Hartenberg parametre.

Der foreligger en fast procedure til udledning af DH parametre. 
Først tildeles der koordinatrammer til robottens led. Nedenstående illustrerer, hvordan dette kan se ud.

\figur{0.50}{DenavitHartenbergFrames.png}{Tildeling af koordinatrammer til robotarm.}{fig:DHFrames}

Rammer tildeles efter visse regler. 

Der startes fra baserammen. Z aksen lægges i leddets rotation. Ved base rammen er x aksens placering vilkårlig. Y aksen lægges således, at koordinatsættes overholder højrehåndsreglen. 
	
For efterfølgende led placeres z aksen ligeledes i leddets rotation. X aksen er her bestemt af, at den skal ligge ortogonalt på foregående leds z akse. Y aksen placeres som før således at højrehåndsreglen overholdes.

Efter tildeling af koordinatrammerne kan Denavit Hartenberg parametrene udledes. Disse består af fire parametre.
	 
\begin{itemize}
\item d - dybden langs det forelående leds z akse.
\item $\theta$ - rotationen omkring det foregående leds z akse fra dette x akse, til det nye leds x akse. 
\item a - længden mellem det foregående leds z akse og det nye leds z akse langs dettes x akse.
\item $\alpha$ - rotationen omkring det nye leds x akse for at bringe z aksen til den ønskede orientering.
\end{itemize}

Figur \ref{fig:DHParams} viser et eksempel på udledning af Denavit Hartenberg parametre. 

\figur{0.50}{DenavitHartenbergParams.png}{Udledning af Denavit Hartenberg parametre.}{fig:DHParams}







