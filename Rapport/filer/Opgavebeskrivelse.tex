\chapter{Opgavebeskrivelse}\label{chap:Opgavebeskrivelse}
I faget ITROB1 er der blevet stillet en opgave om, at skrive et program, som skal kunne styre den mekaniske robotarm også kaldet CrustCrawler. Selve opgaven blev stillet meget fri, og derfor var det op til gruppen, at bestemme hvordan denne skulle løses. De eneste krav til opgaven var at der skulle oprettes en forbindelse mellem roboten og den tilhørende webcam. Dertil skulle der også oprettes to seperate noder, hvilket vil gøre at noget af CrustCrawlerens funktionalitet bliv kørt asynkront. \newline
Det blev bestemt af gruppen, at CrustCrawler skulle kunne flytter klodser fra en position til den inverse position. For at kunne inkludere kameraet (webcam) bliver dette brugt til at finde klodsernes start position. Grundet klodserne har forskellige farver kan kameraet via nogle grænseværdier skelne de forskellige klodser fra hinanden, og derved finde deres positioner. CrustCrawleren får disse koordinater og samler klodsen op. Via inverse kinematik regning bliver den inverse position fundet. CrustCrawleren lægger klodsen på den ny fundne position og retunerer til udgangspunktet. 

\figur{0.5}{CrustcrawlerRobot}{ Robot Figur}{fig:CrustcrawlerRobot}