\chapter{Robot Operating System (ROS)}\label{chap:ROS}
De grundlæggende begreber for ROS implementationen er:
\begin{itemize}  
\item Nodes
\item Messages
\item Topics
\item Services
\end{itemize}
\textbf{Nodes:} Noder er de processer, som udfører en databehandling. For at kunne kommunikerer med mange noder, er det praktisk at bruge P2P kommunikationsmetoden. På den måde kommunikerer noder gennem beskeder, og kan køre asynkront i forhold til hinanden. Derved opdeles arbejdsbyrden fra den enkelt tråd.\newline
\newline
\textbf{Messages:} Messages er en datastruktur, hvor primitive typer,  såsom integers, floating points, booelans, stings osv. er understøttede. Udover disse primitive typer som brugerdefinerede typer også afsendes, så længe disse er registrerede i det kørende ROS system. Disse beskeder bliver benyttet af noderne til at kommunikerer med hinanden. Beskederne kan sammensættes af flere forskellige beskeder, og derved oprette en message queue.\newline
\newline
\textbf{Topic:} Når en node udfører en kommando, sender den en besked afsted. Denne besked bliver så published på et givent topic. Når noden har published en besked, kan en anden node i systemet subcribe på dette Topic. Denne node vil da modtage de afsendte data. Dette er et publisher / subscribe forhold. Grunden til dette forhold mellem noder er, at den node som publisher, og den node som subscriber er ikke afhængige af hinanden, og har derfor en lav kobling. Derved opstår der færre komplikationer mellem disse noder.\newline
Nedenfor ses en figur over hvordan kommunikationen mellem to noder fungerer.
\figur{0.6}{ROStopic_figur}{ROS Topic}{fig:ROStopic_figur}
\textbf{Service:} Publish / subcribe forholdet er en meget fleksibel form for kommunikation også kaldet broadcasting. Denne broadcasting er asynkront, og køre derfor uafhængigt af ''main'' tråden. En service er defineret af et navn og typefast besked. Den ene bruges som en request og den anden bruges som en response. Services anvender srv filer, som er kompileret i source koden ved et ROS-client bibliotek.

Se bilag \vref{app:ROSPrincipper} for principperne bag ROS.