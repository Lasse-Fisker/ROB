\chapter{Robot Operating System (ROS)}\label{chap:ROS}
\section{Robot framework}
Ros står for Robot Operatin System, som er et open-source system. Det kan være svært at skrive software programmer for robotter, da udviklingen af robotter er stigende og de kommer i forskellige former og størrelser. De forskellige robotter kan have varierende hardware, hvilket resulterer i at genbrug af software ikke altid er en mulighed. Derfor har robotforsker/ingeniør igennem tiden udviklet mange forskellige framworks, som kan håndtere de enkelte robotter. Da dette er ekstremt uhensigtsmæssig og kan føre til at skulle omskrive kode igen og igen, har robotingeniør udviklet et framework, som kan håndtere disse udfordringer inden for robot området. ROS frameworket blev udviklet i 2007, og er et samlet produkt af kompromiser og prioriteringer, som blev valgt i desginfasen. Selvom ROS framewroket blev udviklet som en samlet løsning, så har frameworket dets begrænsninger og derfor er det måske ikke det bedste framework til udvikling af software for robotter. Dokumentation ” ROS: an open-source Robot Operating System” mener, at der ikke nødvendigvis findes et framework, som er det bedste for hver enkelt robot. ROS bliver hele tiden forbedret, samtidig med at der kommer flere robotter til, og derfor er det et aldrig færdig framework. \newline
\section{Grundlæggende principper}
Filosofien for ROS og de mest almindelige funktioner er vist på figuren nedenfor.
\figur{1}{ROSfilo_figur}{ROS design kriterier}{fig:ROSfilo_figur}
Frameworket er designet efter nedenstående kriterier. 
\begin{itemize}  
\item Peer-to-peer
\item Tools-based 
\item Multi-lingual
\item Thin
\item Free and Open-Source\ldots 
\end{itemize}
