\chapter{Koordinatsystem konvertering}\label{app:KoordinatsystemKonvertering}
For at identificere hvor mange pixels, der er pr. enhed i CrustCrawlerens koordinatsystem, udføres en test, hvor klodser sættes i et kendte punkter i CrustCrawlerens koordinatsystem.
Dette gøres ved at placere robotten i et koordinat f.eks. (10,10) og placere en klods netop der.
Derefter placeres robotten et sted uden for billedet hvorefter webcamet tager et billede af den placerede klods.
På dette billede udføres vision funktionerne  hvilke returnerer et punkt (række og kolonne) i billedet koordinatsystem.



Nedenstående er en tabel over resultaterne:
\begin{table}[H]
\centering
\begin{tabular}{c|c|c}
CrustCrawler koordinater (x,y)	&	Billede koordinater (kolonne,række)	& Forhold\\
\hline
(10,-10)	&	(206,95)		&	(8.9,9.5)\\
(10,-15)	&	(203,131)	&	(9.2,8.7)\\
(10,-20)	&	(204,185)	&	(9.1,9.3)\\
(15,-15)	&	(165,139)	&	(8.7,9.2)\\
(15,-10)	&	(158,95)		&	(9.1,9.5)\\
(15,-20)	&	(159,176)	&	(9.1,8.8)\\
(20,-20)	&	(110,178)	&	(9.3,8.9)\\
(20,-10)	&	(117,85)		&	(8.9,8.5)\\
(20,-15)	&	(116,132)	&	(9.0,8.8)\\
\end{tabular}	
\caption{Grænseværdier til identifikation af de fire farver, blå, rød, grøn og gul. MATLAB range 0-1.}
\end{table}

Ud fra resultaterne i tabellen herover udregnes middelværdien for antal pixels pr. enheder i CrustCrawlerens koordinatsystem i x og y retningen:
\begin{equation}
	mid_x=\frac{8.9+9.2+9.1+8.7+9.1+9.1+9.3+8.9+9.0}{9}=9.0
\end{equation}

\begin{equation}
	mid_y=\frac{9.5+8.7+9.3+9.2+9.5+8.8+8.9+8.5+8.8}{9}=9.0
\end{equation}

Der er derfor 9 pixels pr. enhed i CrustCrawlerens koordinatsystem.