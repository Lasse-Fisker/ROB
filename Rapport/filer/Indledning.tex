\chapter{Indledning}\label{chap:Indledning}
I dagens industrielle verden anvendes robotter i høj grad.
De letter arbejdsbyrden for den almindelige arbejder, og opnår en hurtigere arbejdsgang.
Robotterne er med til at fremme arbejdsprocessen, da virksomhederne selvfølgelig er interesserede i at holde omkostningerne lave.
I industrielt øjemed er det ofte billigere at have en robot, der kan arbejde døgnet rundt, end et menneske, som har krav på pauser og overensstemmelser og som ikke kan arbejde i farlige miljøer.
Robotterne kan for eksempel bruges til at overvåge andre systemer, og give feedback på de data, som den har adgang til.
Robotterne er ofte meget effektive og ikke mindst præcise i deres arbejde, og kan derfor spare virksomheder for menneskelige fejl.
Mennesker flytter sig i stigende grad fra at varetage den fysiske del af produktionen, til at håndtere det kreative og innovative aspekt.
Robotter og automatiserede arbejdsgange bliver i højere og højere grad en fast inkorporeret del af en moderne produktionsprocess, og de er kommet for at blive.

%%%%%%%%%% Opgavebeskrivelse %%%%%%%%%%
\section{Opgavebeskrivelse}\label{sec:Opgavebeskrivelse}
Som afslutning på faget ITROB1 udfærdiges et projekt.
Rammerne omkring dette projekt var, at \textit{CrustCrawler AX 12 A Smart Robotic Arm}, se figur \ref{fig:CrustCrawlerRobot}, samt et vision system, med min. en sensor, skulle anvendes. Udover dette var det eneste formelle krav, at systemet skulle bestå af minimum to ROS noder. 
Projektets emne var frit for gruppen, så længe ovenstående krav blev opfyldt. 
Ved projektets afslutning antages det, at en fungerende prototype på systemet er være realiseret.
Udover dette fungerer denne rapport som dokumentation af projektet.
\figur{0.30}{CrustcrawlerRobot}{CrustCrawler AX 12 A Smart Robotic Arm}{fig:CrustCrawlerRobot}