\chapter{Indledning}\label{chap:Indledning}
I den industrielle verden i dag anvendes robotter i høj grad.
Robotterne bruges til at lette arbejdsbyrden for den almindelige arbejder, og for at opnå en hurtigere arbejdsgang.
Robotterne er simpelthen med til at fremme arbejdsprocessen, da virksomhederne er interesseret i at holde omkostningerne lave.
For indrustriverdenen er det billigere at have en robot, der kan arbejde fireogtyve timer i døgnet end et menneske, som har krav på en pause en gang i mellem.
Robotterne kan for eksempel bruges til, at overvåge andre systemer, og give feedback på de data, som den har adgang til.
Robotterne er ofte meget effektive og ikke mindst præcise i deres arbejde, og kan derfor spare virksomheder for menneskelige fejl.
Mennesker flytter sig mere og mere fra at skulle være i den fysiske del af en produktion, til at være i den kreative del, altså udviklingsledet.
Robotterne overtager stille og roligt den fysiske del, og de er kommet for at blive.

%%%%%%%%%% Opgavebeskrivelse %%%%%%%%%%
\section{Opgavebeskrivelse}\label{sec:Opgavebeskrivelse}
Som afslutning på faget ITROB1 udfærdiges et projekt.
Rammerne omkring dette projekt var, at CrustCrawler AX 12 A Smart Robotic Arm, se figur \ref{fig:CrustCrawlerRobot}, samt et vision system, med min. en sensor, skulle anvendes. Udover dette var det eneste formelle krav, at systemet skulle bestå af minimum to ROS noder. 
Projektets emne var frit for gruppen, så længe ovenstående krav blev opfyldt. 
Ved projektets afslutning forventes det, at en fungerende prototype på systemet er realiseret.
Udover dette fungerer denne rapport som dokumentation af projektet.
\figur{0.37}{CrustcrawlerRobot}{CrustCrawler AX 12 A Smart Robotic Arm}{fig:CrustCrawlerRobot}