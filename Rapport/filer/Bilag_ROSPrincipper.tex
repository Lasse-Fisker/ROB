\chapter{ROS principper}\label{app:ROSPrincipper}
%%%%%%%%%% ROS framework %%%%%%%%%%
\section{ROS framework}
Robot Operating System, herefter benævnt ROS, er et open-source framework til styring af robotter.
Det kan være svært at skrive software for robotter, da udviklingen af robotter er stigende og de kommer i forskellige former og størrelser.
De forskellige robotter kan have varierende hardware, hvilket resulterer i, at genbrug af software ikke altid er en mulighed.
Derfor har robotforskerer-og ingeniører historisk set udviklet mange forskellige frameworks til håndtering af de enkelte robotter.
Dette var ekstremt uhensigtsmæssig, da det var svært at genbruge kode og svært for programmører at gå fra én robot til en anden uden at lære nye frameworks.
ROS frameworket blev udviklet i 2007 som en løsning på dette problem. Det er et produkt af kompromiser og prioriteringer, som blev valgt i designfasen.
Selvom ROS framewroket blev udviklet som en samlet løsning, så har frameworket dets begrænsninger og er derfor ikke altid det optimale valg af software til robotter.
Dokumentation ''ROS: an open-source Robot Operating System'' mener, at der ikke nødvendigvis findes et framework, som er det bedste for hver enkelt robot.
ROS bliver hele tiden forbedret, samtidig med, at der kommer flere robotter til, og derfor er det et framework der aldrig bliver færdigt.

%%%%%%%%%% Grundlæggende ROS principper %%%%%%%%%%
\section{Grundlæggende ROS principper}
Filosofien for ROS og de mest almindelige funktioner er vist på figuren nedenfor.
\figur{1}{ROSfilo_figur}{ROS design kriterier}{fig:ROSfilo_figur}
Frameworket er designet efter følgende kriterier:
\begin{itemize}  
\item Peer-to-peer
\item Tools-based 
\item Multi-lingual
\item Thin
\item Free and Open-Source\ldots 
\end{itemize}
\textbf{Peer to peer (P2P):}
Et ROS system anvender en række processer på et bredt antal af forskellige hosts, som er forbundet runtime i en peer-to-peer topologi.
Disse processer består af noder, hvor hver node udfører én specifik handling. På store industrielle robotter, er der typisk onboard maskiner som er forbundet med ethernet.
Dette netværk er ''bridget'', med mulighed for at kommunikere trådløst til en offboard maskine, der anvender et vision system eller voice recognition.
ROS P2P har ikke en central server, og der kan det undgås, at on- og offboard maskiner vil medføre stor data trafik, og dermed gøre den trådløse forbindelse langsom pga. af de informationer, der bliver pushed og lageret i subnettet.
På figur \ref{fig:ROS_konfiguration_figur} ses det grafisk, hvordan P2P fungere. 
\figur{1}{ROS_konfiguration_figur}{ROS netværks konfiguration}{fig:ROS_konfiguration_figur}
\textbf{Tool-based:} For at kunne håndtere kompleksiteten af ROS, er der valgt at bruge et microkernel design.
Designet er opbygget således, at et stort antal af mindre værktøjselementer er brugt til ROS komponenterne.
Dette design er blevet valgt frem for at bruge en monolitisk udvikling og runtime miljø.
Disse små værktøjer kan alt fra at navigere i source koden til at sætte konfigurations paramenterne osv.\\
\newline
\textbf{Multi-lingual:} Multi-lingual gør, at ROS er et programmeringssprog, som supporterer flere forskellige sprog.
Disse sprog er i skrivende stund C++, Python, Octave, LISP.
At ROS supporter disse forskellige programmeringssprog gør, at ROS i sig selv er let at anvende for de fleste programmører, og derfor appelere et bredere publikum.
For at supportere cross-language, anvender ROS et simpel sprogneutralt interface, \textit{IDL}, til at beskrive beskeder, der sendes gennem modulerne.
\textit{IDL} anvender en kort tekstfil til at beskrive felterne af hver enkelt besked, og tillader en sammensætning af beskeder.
Dette ses på figur \ref{fig:IDEmessage_figur}.
\figur{0.6}{IDEmessage_figur}{IDE message file}{fig:IDEmessage_figur}
ROS udgør altså en platform, hvor der er mulighed for at blande de fornævnte programmeringssprog på kryds og tværs som der ønskes uden at kompromittere den samlede funktionalitet.\\
\newline
\textbf{Thin:} Mange projekter inden for robotteknologi består af mange forskellige drivers og algoritmer, som kan bruges uden for det egentlige projekt.
Derfor er ROS designet til at være så \textit{tyndt} som overhovedet muligt.
Dette skyldes, at der ikke skal være for mange begrænsninger, som kan udelukke andre robot frameworks fra at arbejde sammen med ROS.
ROS er derfor let at integrere med andre robotframeworks.
Når ROS importeres i ens projekt, anvendes forskellige imports af enkeltstående biblioteker, som har en minimal afhængighed til ROS.
Der genbruges også kode fra eksisterende open-source projekter, som for eksempel Vision algoritmen fra OpenCV.\\
\newline
\textbf{Free and Open-Source:}
ROS er frit tilgængeligt, og kan benyttes af alle, der ønsker det.
Dog har ROS værktøjer, der er lukkede for den almene bruger, og kræver en licens for at få adgang dertil.