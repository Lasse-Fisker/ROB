\chapter{Systembeskrivelse}\label{chap:Systembeskrivelse}
Nedenstående beskriver et system, Pattern Mirroring Device, herefter benævnt PMD, bestående af en
robotarm, der anvendes til at konstruere et spejlbillede af et eksisterende mønster bestående af
farvede klodser.

Før spejling klargøres et mønster bestående af klodser med klare, ensartede, farver på det
dertilhørende opstillingsareal (A).
PMD vil derefter påbegynde duplikeringen af det fremlagte mønster, blot spejlvendt, på det
tilstødende duplikeringsareal (B).
Ved færdig spejling vil PMD bevæge sig til en neutral position og påbegynde det hardcodede
afslutningssignal. Brugerens hånd placeres, med håndfladen opad, i midten af PMD’s
arbejdsområde. PMD vil bevæge sig mod hånden. Ved kontakt
mellem bruger of PMD signalerer korrekt spejlning. PMD returnerer herefter til udgangspositionen.
\figur{1}{Opgavebeskrivelse_Robotfigur}{Systemillustration}{fig:Systemillustration}

%%%%%%%%%% Hardware %%%%%%%%%%
\section{Hardware}\label{sec:Hardware}
Systemet indeholder vision recognition i form af et fast monteret kamera, der anvendes til at
analysere det eksisterende mønster og til at udrssegne de påkrævede koordinater, som PMD skal
navigere til. 

Udover dette anvendes en strain gauger til at detektere, hvorvidt et objekt er holdt fast af robotten 
ved at måle det tryk, som denne udøver på den givne overflade. strain gauge er tilsluttet en arduino microcontoller.

Robottens eksisterende aktuator anvendes til at manipulere de påkrævede objekter.
%%%%%%%%%% Software %%%%%%%%%%
\section{Software}\label{sec:Software}
Strain gauge bliver implementeret i en arduino microcontroller. For at denne kan kommunikere med resten af system er
biblioteket ROS Seriel blevet brugt, til at lave en publisher. sproget er adruinos subset af C.

Programmet til at controllere robotten er skrevet i python hvor ROS PY biblioteket er blevet brugt. Robotten bliver styret fra en virtuelmaskine i et linux miljø.