\chapter{Python Kode}\label{app:PythonKode}
%%%%%%%%%% main.py %%%%%%%%%%
\section{main.py}\label{sec:Main}
\begin{lstlisting}[language=Python]
#!/usr/bin/env python

import rospy
import actionlib
from control_msgs.msg import FollowJointTrajectoryAction
from control_msgs.msg import FollowJointTrajectoryFeedback
from control_msgs.msg import FollowJointTrajectoryResult
from control_msgs.msg import FollowJointTrajectoryGoal
from trajectory_msgs.msg import JointTrajectoryPoint
from trajectory_msgs.msg import JointTrajectory
import math
import time
from std_msgs.msg import String

from InvRobot import *
from GetCoordinates import *

visionCoor = []

def main():
        
    Jobactive = True
    Job = False
    
    while Jobactive == True:
        xy = returnvisionCoor()
        print "XY: "
        print xy
        if len(xy) > 0:
            mirrorCube(xy)
            Job = True
        else:
            Jobactive = False
            if Job == True:
                RobotHandshake()
                print "Jobs done"
            else:
                print "There wasnt any job "                
if __name__ == "__main__":
    setUpVisonCoordinates()
    setupGrabberPressureSensor()
    top = invkin([0,0, 54.1])
    RobotDo(top,0,0)
    print 'Main started: '
    print '______________________'
    main()
\end{lstlisting}

\newpage
%%%%%%%%%% Vision.py %%%%%%%%%%
\section{Vision.py}\label{sec:VisionNode}
\begin{lstlisting}[language=Python]
#!/usr/bin/env python

import rospy
from findBricks import *
from std_msgs.msg import String

def VisionPublisher():
    
    pub = rospy.Publisher('Coordinates', String, queue_size=10)
    rospy.init_node('VisionPublisher', anonymous=True)
    rate = rospy.Rate(1) # 1hz
    while not rospy.is_shutdown():
        try:
            bricks = find_brick_centers()
            
            if len(bricks) != 0:        
                coords_str = "%f,%f"%(bricks[0], bricks[1])
            else:
                coords_str = "%f,%f"%(0, 0)          
            pub.publish(coords_str)
            
            print("Test af coords_str")
            print(coords_str)
        except:
            print "Error"
        rate.sleep()

if __name__ == '__main__':
    try:
        VisionPublisher()
    except rospy.ROSInterruptException:
        pass
\end{lstlisting}

\newpage
%%%%%%%%%% findBricks.py %%%%%%%%%%
\section{findBricks.py}\label{sec:VisionFuntioner}
\begin{lstlisting}[language=Python]
#!/usr/bin/env python

import cv2
import urllib
import numpy as np
import math

def get_from_webcam():
    print "try fetch from webcam..."
    stream=urllib.urlopen('http://192.168.0.20/image/jpeg.cgi')
    bytes=''
    bytes+=stream.read(64500)
    a = bytes.find('\xff\xd8')
    b = bytes.find('\xff\xd9')

    if a != -1 and b != -1:
        jpg = bytes[a:b+2]
        bytes= bytes[b+2:]
        i = cv2.imdecode(np.fromstring(jpg, dtype=np.uint8),cv2.CV_LOAD_IMAGE_COLOR)
        i_crop = i[55:350, 300:610]
        return i_crop
    else:
        print "did not receive image, try increasing the buffer size in line 13:"

def extract_single_color_range(image,hsv,lower,upper):
    if len(lower) == 2 and len(upper) == 2:
        mask0 = cv2.inRange(hsv, lower[0], upper[0])
        mask1 = cv2.inRange(hsv, lower[1], upper[1])
        mask = mask0+mask1
    else:
        mask = cv2.inRange(hsv, lower, upper)
    res = cv2.bitwise_and(image,image, mask= mask)
    return res

def threshold_image(image):
    ret,th1 = cv2.threshold(image,50,255,cv2.THRESH_BINARY)
    resdi = cv2.dilate(th1,np.ones((3,3),np.uint8))
    closing = cv2.morphologyEx(resdi, cv2.MORPH_CLOSE,np.ones((5,5),np.uint8))
    return closing

def contours(image):
    imgray = cv2.cvtColor(image,cv2.COLOR_BGR2GRAY)
    contours, hierarchy = cv2.findContours(imgray,cv2.RETR_TREE,cv2.CHAIN_APPROX_SIMPLE)
    return contours

def get_centers(contours):
    centers = []
    for cnt in contours:
        epsilon = 0.1*cv2.arcLength(cnt,True)
        approx = cv2.approxPolyDP(cnt,epsilon,True)
        area = cv2.contourArea(approx)
        
        if area > 500:
            moments = cv2.moments(cnt)
            centers.append((int(moments['m10']/moments['m00']), int(moments['m01']/moments['m00'])))
    
    return centers

def find_brick_centers():    
    lower_blue = np.array([92,76,103])
    upper_blue = np.array([141,255,255])
    
    lower_green = np.array([36,76,0])
    upper_green = np.array([74,255,255])
    
    lower_yellow = np.array([21,76,103])
    upper_yellow = np.array([38,255,255])
    
    lower_red = np.array([np.array([0,76,103]),np.array([161,76,103])])
    upper_red = np.array([np.array([14,255,255]),np.array([179,255,255])])

    image = get_from_webcam()
    hsv = cv2.cvtColor(image, cv2.COLOR_BGR2HSV)
    
    single_color_img_blue = extract_single_color_range(image,hsv,lower_blue,upper_blue)
    single_color_img_green = extract_single_color_range(image,hsv,lower_green,upper_green)
    single_color_img_yellow = extract_single_color_range(image,hsv,lower_yellow,upper_yellow)
    single_color_img_red = extract_single_color_range(image,hsv,lower_red,upper_red)
        
    single_channel_blue = threshold_image(single_color_img_blue)
    single_channel_green = threshold_image(single_color_img_green)
    single_channel_yellow = threshold_image(single_color_img_yellow)
    single_channel_red = threshold_image(single_color_img_red)
    
    cont_blue = contours(single_channel_blue)
    cont_green = contours(single_channel_green)
    cont_yellow = contours(single_channel_yellow)
    cont_red = contours(single_channel_red)
    
    centers_blue = get_centers(cont_blue)
    centers_green = get_centers(cont_green)
    centers_yellow = get_centers(cont_yellow)
    centers_red = get_centers(cont_red)
    
    centers_cm = []
    for c in centers_blue:
        x_koor = float((295.0-c[1])/9.0)
        y_koor = float(-(c[0])/9.0)
        centers_cm.append(x_koor)
        centers_cm.append(y_koor)
        
    for c in centers_green:
        x_koor = float((295.0-c[1])/9.0)
        y_koor = float(-(c[0])/9.0)
        centers_cm.append(x_koor)
        centers_cm.append(y_koor)
        
    for c in centers_yellow:
        x_koor = float((295.0-c[1])/9.0)
        y_koor = float(-(c[0])/9.0)
        centers_cm.append(x_koor)
        centers_cm.append(y_koor)
    
    for c in centers_red:
        x_koor = float((295.0-c[1])/9.0)
        y_koor = float(-(c[0])/9.0)
        centers_cm.append(x_koor)
        centers_cm.append(y_koor)
                          
    return centers_cm
\end{lstlisting}

\newpage
%%%%%%%%%% InvRobot.py %%%%%%%%%%
\section{InvRobot.py}\label{sec:InvRobot}
\begin{lstlisting}[language=Python]
#!/usr/bin/env python


import rospy
import actionlib
from control_msgs.msg import FollowJointTrajectoryAction
from control_msgs.msg import FollowJointTrajectoryFeedback
from control_msgs.msg import FollowJointTrajectoryResult
from control_msgs.msg import FollowJointTrajectoryGoal
from trajectory_msgs.msg import JointTrajectoryPoint
from trajectory_msgs.msg import JointTrajectory
import math
import time
from std_msgs.msg import Int32

grabberPressure = 0

def invkin(xyz):
    """
    Python implementation of the the inverse kinematics for the crustcrawler
    Input: xyz position
    Output: Angels for each joint: q1,q2,q3,q4,q5
    
    You might adjust parameters (d1,a1,a2,d4).
    The robot model shown in rviz can be adjusted accordingly by editing au_crustcrawler_ax12.urdf
    """

    d1 = 16.8; # cm (height of 2nd joint)
    a1 = 0.0; # (distance along "y-axis" to 2nd joint)
    a2 = 17.31; # (distance between 2nd and 3rd joints)
    d4 = 20; # (distance from 3rd joint to gripper center - all inclusive, ie. also 4th joint)
    
    x1 = xyz[0]
    y1 = xyz[1]
    z1 = xyz[2]

    q1 = math.atan2(y1, x1)

    # Calculate q2 and q3
    r2 = math.pow((x1 - a1 * math.cos(q1)),2) + math.pow((y1 - a1 * math.sin(q1)),2)
    s = (z1 - d1)
    D = (r2 + math.pow(s,2) - math.pow(a2,2) - math.pow(d4,2)) / (2 * a2 * d4)
    
    q3 = math.atan2(-math.sqrt(1 - math.pow(D,2)), D)
    q2 = math.atan2(s, math.sqrt(r2)) - math.atan2(d4 * math.sin(q3), a2 + d4 * math.cos(q3))-(math.pi/2)

    #q4 = 0# not consider rotation so far..
    
    #print '______________'
    #print 'r2 = ' , r2
    #print 's = ' , s
    #print 'D = ' , D
    #print 'q1 = ', q1, '::' , ' q2 = ' , q2, '::' , ' q3 = ' , q3
    #print '______________'

    return q1,q2,q3

class RobotExecute:

    N_JOINTS = 5

    #constructur
    def __init__(self,server_name, angles, rotate, gripper):
        self.client = actionlib.SimpleActionClient(server_name, FollowJointTrajectoryAction)

        self.joint_positions = []
        self.names =["joint1",
                        "joint2",
                        "joint3",
                        "joint4",
                        "gripper"
                        ]
        # the list of xyz points we want to plan
        joint_positions = [
        [angles[0], angles[1], angles[2], rotate, gripper]
        ]

        # initial duration
        dur = rospy.Duration(1)

        # construct a list of joint positions
        for p in joint_positions:
            jtp = JointTrajectoryPoint(positions=p,velocities=[0.5]*self.N_JOINTS, time_from_start=dur)
            dur += rospy.Duration(5)
            self.joint_positions.append(jtp)
        
        # create joint trajectory
        self.jt = JointTrajectory(joint_names=self.names, points=self.joint_positions)

        # create goal
        self.goal = FollowJointTrajectoryGoal( trajectory=self.jt, goal_time_tolerance=dur+rospy.Duration(2) )

    def send_command(self):
        self.client.wait_for_server()
        #print self.goal
        self.client.send_goal(self.goal)

        self.client.wait_for_result()
        #print self.client.get_result()


def RobotDo(angles, rotate, gripper):   
    node = RobotExecute("/arm_controller/follow_joint_trajectory", angles, rotate, gripper)
    
    node.send_command()

def mirrorCube(xy):

    air1 = 15
    air2 = 30
    table = 6
    grabber_pos = 0
    not_rotated = 0
    rotated = 1.5
    top = invkin([0, 0, 54.1])
    time1 = 1
    time2 = 0.5

    #in air
    RobotDo(invkin([xy[0], xy[1], air2]), not_rotated, grabber_pos)
    time.sleep(time1)

    #to table
    RobotDo(invkin([xy[0], xy[1], table]), not_rotated, grabber_pos)
    time.sleep(time1)

    #grab brick. While pressure is low enough, increase grabber position until brick is secured.
    while grabberPressure < 600:
        print "grabber pressure is: %d" % grabberPressure
        grabber_pos += 0.4
        RobotDo(invkin([xy[0], xy[1], table]), not_rotated, grabber_pos)

    #rotate - to air low
    RobotDo(invkin([xy[0], xy[1], air1]), rotated, grabber_pos)
    time.sleep(time2)

    #to air 
    RobotDo(invkin([xy[0], xy[1], air2]), rotated, grabber_pos)
    time.sleep(time1)

    #to mirrored X Y
    RobotDo(invkin([xy[0], -xy[1], air2]), rotated, grabber_pos)
    time.sleep(time2)

    # lower
    RobotDo(invkin([xy[0], -xy[1], air1]), rotated, grabber_pos)
    time.sleep(time1)

    #table un rotate
    RobotDo(invkin([xy[0], -xy[1], table]), not_rotated, grabber_pos)
    time.sleep(time1)

    #release
    grabber_pos = 0
    RobotDo(invkin([xy[0], -xy[1], table]), not_rotated, grabber_pos)
    time.sleep(time1)

    # to air
    RobotDo(invkin([xy[0], -xy[1], air1]), not_rotated, grabber_pos)
    time.sleep(time2)

    # higher air
    RobotDo(invkin([xy[0], -xy[1], air2]), not_rotated, grabber_pos)
    RobotDo(top, not_rotated, grabber_pos)
    time.sleep(2)

def RobotHandshake():

    x = 28
    y = 0
    z = 30
    not_rotated = 0
    not_grapped = 0
    top = invkin([0, 0, 54.1])
    
    RobotDo(invkin([x, y, z]),not_rotated,not_grapped)

    while grabberPressure < 700:
        pass

    RobotDo(top, not_rotated, not_grapped)

# setup subscriber for pressure sensor
def setupGrabberPressureSensor():
    rospy.Subscriber("grabber_pressure", Int32, handleReadPressure)

# callback method for pressure sensor
def handleReadPressure(val):
    global grabberPressure
    grabberPressure = val.data

    #debug
\end{lstlisting}

%%%%%%%%%% GetCoordinates.py %%%%%%%%%%
\section{GetCoordinates.py}\label{sec:GetCoordinates}
\begin{lstlisting}[language=C]
#!/usr/bin/env python

import rospy
import actionlib
from control_msgs.msg import FollowJointTrajectoryAction
from control_msgs.msg import FollowJointTrajectoryFeedback
from control_msgs.msg import FollowJointTrajectoryResult
from control_msgs.msg import FollowJointTrajectoryGoal
from trajectory_msgs.msg import JointTrajectoryPoint
from trajectory_msgs.msg import JointTrajectory
import math
import time
from std_msgs.msg import String

from findBricks import *

visionCoor = []

def getCoordinates(coord):    
    global visionCoor
    
    coord_str = str(coord)
    coor = coord_str.replace("data: ","")
    visionCoor = [float(s) for s in coor.split(',')]
    
    if visionCoor[0] == 0 and visionCoor[1] == 0:
        visionCoor = []

# Subcribing on Coordinates node
def setUpVisonCoordinates():
    rospy.init_node("InvRobot")
    rospy.Subscriber("Coordinates", String, getCoordinates)

# Return the vision coordinates
def returnvisionCoor():
    return visionCoor
\end{lstlisting}
\newpage
%%%%%%%%%% arduino_strain_gauge.ino %%%%%%%%%%
\section{arduino\_strain\_gauge.ino}\label{sec:StrainGauge}
\begin{lstlisting}[language=C]
#include <ros.h>
#include <std_msgs/Bool.h>

ros::NodeHandle  nh;
std_msgs::Bool grabbedSomethingMsg;

boolean grabbedSomething = false;
boolean shouldCheck = true;

ros::Publisher grabber_strain_gauge_publisher("grabbed_something", &grabbedSomethingMsg);
int strainGaugePin = A0;

void handleCheckPressureCb(const std_msgs::Bool& msg) {  
      shouldCheck = msg.data;
}

ros::Subscriber<std_msgs::Bool> grabber_strain_gauge_subscriber("grabber_check_pressure", &handleCheckPressureCb);

void setup()
{
  nh.initNode();
  nh.advertise(grabber_strain_gauge_publisher);
  
  //for debug
  Serial.begin(57600);
  
  
}

void loop()
{
    if(shouldCheck){
        checkStrain();
    }
    
    
    nh.spinOnce();
  
    delay(10);
}

void checkStrain() {
  
    //read pressure, publish if high enough. TODO: test values
    if(analogRead(strainGaugePin) > 400) {
        publish(true);
    } else {
        publish(false);
    }
}

void publish(boolean val) {
    grabbedSomethingMsg.data = val;
    grabber_strain_gauge_publisher.publish( &grabbedSomethingMsg );
}
\end{lstlisting}