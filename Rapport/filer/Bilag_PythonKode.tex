\chapter{Python Kode}\label{app:PythonKode}
\section{Main}\label{sec:Main}
\begin{lstlisting}[language=Python]
\end{lstlisting}
\section{Vision Node}\label{sec:VisionNode}
\begin{lstlisting}[language=Python]
#!/usr/bin/env python

import rospy
from findBricks import *
from std_msgs.msg import String

def VisionPublisher():
    
    pub = rospy.Publisher('Coordinates', String, queue_size=10)
    rospy.init_node('VisionPublisher', anonymous=True)
    rate = rospy.Rate(1) # 10hz
    while not rospy.is_shutdown():
        try:
            bricks = find_brick_centers()
            
            if len(bricks) != 0:        
                coords_str = "%f,%f"%(bricks[0], bricks[1])
            else:
                coords_str = "%f,%f"%(0, 0)          
            pub.publish(coords_str)
            
            print("Test af coords_str")
            print(coords_str)
        except:
            print "Error"
        rate.sleep()

if __name__ == '__main__':
    try:
        VisionPublisher()
    except rospy.ROSInterruptException:
        pass
\end{lstlisting}
\section{Vision funktioner}\label{sec:VisionFuntioner}
\begin{lstlisting}[language=Python]
#!/usr/bin/env python
import cv2
import urllib
import numpy as np
import math

def get_from_webcam():
    """
    Fetches an image from the webcam
    """
    print "try fetch from webcam..."
    stream=urllib.urlopen('http://192.168.0.20/image/jpeg.cgi')
    bytes=''
    bytes+=stream.read(64500)
    a = bytes.find('\xff\xd8')
    b = bytes.find('\xff\xd9')

    if a != -1 and b != -1:
        jpg = bytes[a:b+2]
        bytes= bytes[b+2:]
        i = cv2.imdecode(np.fromstring(jpg, dtype=np.uint8),cv2.CV_LOAD_IMAGE_COLOR)
        i_crop = i[55:350, 300:610]
        return i_crop
    else:
        print "did not receive image, try increasing the buffer size in line 13:"

def extract_single_color_range(image,hsv,lower,upper):
    """
    Calculates a mask for which all pixels within the specified range is set to 1
    the ands this mask with the provided image such that color information is
    still present, but only for the specified range
    """
    if len(lower) == 2 and len(upper) == 2:
        mask0 = cv2.inRange(hsv, lower[0], upper[0])
        mask1 = cv2.inRange(hsv, lower[1], upper[1])
        mask = mask0+mask1
    else:
        mask = cv2.inRange(hsv, lower, upper)
    res = cv2.bitwise_and(image,image, mask= mask)
    return res

def threshold_image(image):
    """
    Thresholds the image within the desired range and then dilates with a 3x3 matrix
    such that small holes are filled. Afterwards the 'blobs' are closed using a
    combination of dilate and erode
    """
    ret,th1 = cv2.threshold(image,50,255,cv2.THRESH_BINARY)
    resdi = cv2.dilate(th1,np.ones((3,3),np.uint8))
    closing = cv2.morphologyEx(resdi, cv2.MORPH_CLOSE,np.ones((5,5),np.uint8))
    
    return closing

def contours(image):
    """
    Extract the contours of the image by first converting it to grayscale and then
    call findContours
    """
    imgray = cv2.cvtColor(image,cv2.COLOR_BGR2GRAY)
    contours, hierarchy = cv2.findContours(imgray,cv2.RETR_TREE,cv2.CHAIN_APPROX_SIMPLE)

    return contours

def get_centers(contours):
    """
    For each contour in contours
        approximate the contours such that small variations are removed
        calulate the area of the contour
        if the area is within the desired range we append the box points to the
        bricks.
    """
    centers = []
    for cnt in contours:
        epsilon = 0.1*cv2.arcLength(cnt,True)
        approx = cv2.approxPolyDP(cnt,epsilon,True)
        area = cv2.contourArea(approx)
        
        if area > 500:
            moments = cv2.moments(cnt)
            centers.append((int(moments['m10']/moments['m00']), int(moments['m01']/moments['m00'])))
    
    return centers
    

def find_brick_centers():    
    lower_blue = np.array([92,76,103])
    upper_blue = np.array([141,255,255])
    
    lower_green = np.array([36,76,0])
    upper_green = np.array([74,255,255])
    
    lower_yellow = np.array([21,76,103])
    upper_yellow = np.array([38,255,255])
    
    lower_red = np.array([np.array([0,76,103]),np.array([161,76,103])])
    upper_red = np.array([np.array([14,255,255]),np.array([179,255,255])])

    image = get_from_webcam()
    hsv = cv2.cvtColor(image, cv2.COLOR_BGR2HSV)
    
    single_color_img_blue = extract_single_color_range(image,hsv,lower_blue,upper_blue)
    single_color_img_green = extract_single_color_range(image,hsv,lower_green,upper_green)
    single_color_img_yellow = extract_single_color_range(image,hsv,lower_yellow,upper_yellow)
    single_color_img_red = extract_single_color_range(image,hsv,lower_red,upper_red)
        
    single_channel_blue = threshold_image(single_color_img_blue)
    single_channel_green = threshold_image(single_color_img_green)
    single_channel_yellow = threshold_image(single_color_img_yellow)
    single_channel_red = threshold_image(single_color_img_red)
    
    cont_blue = contours(single_channel_blue)
    cont_green = contours(single_channel_green)
    cont_yellow = contours(single_channel_yellow)
    cont_red = contours(single_channel_red)
    
    centers_blue = get_centers(cont_blue)
    centers_green = get_centers(cont_green)
    centers_yellow = get_centers(cont_yellow)
    centers_red = get_centers(cont_red)
    
    centers_cm = []
    for c in centers_blue:
        x_koor = float((295.0-c[1])/9.0)
        y_koor = float(-(c[0])/9.0)
        centers_cm.append(x_koor)
        centers_cm.append(y_koor)
        
    for c in centers_green:
        x_koor = float((295.0-c[1])/9.0)
        y_koor = float(-(c[0])/9.0)
        centers_cm.append(x_koor)
        centers_cm.append(y_koor)
        
    for c in centers_yellow:
        x_koor = float((295.0-c[1])/9.0)
        y_koor = float(-(c[0])/9.0)
        centers_cm.append(x_koor)
        centers_cm.append(y_koor)
    
    for c in centers_red:
        x_koor = float((295.0-c[1])/9.0)
        y_koor = float(-(c[0])/9.0)
        centers_cm.append(x_koor)
        centers_cm.append(y_koor)
                          
    return centers_cm
\end{lstlisting}
\section{Inverse Robot}\label{sec:InvRobot}
\begin{lstlisting}[language=Python]
\end{lstlisting}
\section{Strain Gauge}\label{sec:StrainGauge}
\begin{lstlisting}[language=C]
\end{lstlisting}